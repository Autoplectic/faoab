

\DeclareMathOperator*{\argmin}{argmin}
\DeclareMathOperator*{\argmax}{argmax}

% Abbreviations from CMPPSS:

\newcommand{\eM}     {\mbox{$\epsilon$-machine}}
\newcommand{\eMs}    {\mbox{$\epsilon$-machines}}
\newcommand{\EM}     {\mbox{$\epsilon$-Machine}}
\newcommand{\EMs}    {\mbox{$\epsilon$-Machines}}
\newcommand{\eT}     {\mbox{$\epsilon$-transducer}}
\newcommand{\eTs}    {\mbox{$\epsilon$-transducers}}
\newcommand{\ET}     {\mbox{$\epsilon$-Transducer}}

% cryptic
\newcommand{\order}[1]{order\protect\nobreakdash-$#1$}
\newcommand{\Order}[1]{Order\protect\nobreakdash-$#1$}
\newcommand{\cryptic}[1]{$#1$\protect\nobreakdash-cryptic}
\newcommand{\Cryptic}[1]{$#1$\protect\nobreakdash-Cryptic}
\newcommand{\crypticity}[1]{$#1$\protect\nobreakdash-crypticity}
\newcommand{\Crypticity}[1]{$#1$\protect\nobreakdash-Crypticity}

\newcommand{\ETs}    {\mbox{$\epsilon$-Transducers}}

% Processes and sequences

\newcommand{\Process}{\mathcal{P}}

\newcommand{\ProbMach}{\Prob_{\mathrm{M}}}
\newcommand{\Lmax}   { {L_{\mathrm{max}}}}
\newcommand{\MeasAlphabet}	{\mathcal{A}}
% Original symbol
%\newcommand{\MeasSymbol}   { {S} }
%\newcommand{\meassymbol}   { {s} }
% New symbol
\newcommand{\MeasSymbol}   { {X} }
\newcommand{\meassymbol}   { {x} }
\newcommand{\BiInfinity}	{ \smash{\overleftrightarrow {\MeasSymbol}} }
\newcommand{\biinfinity}	{ \smash{\overleftrightarrow {\meassymbol}} }
\newcommand{\Past}	{ \smash{\overleftarrow {\MeasSymbol}} }
\newcommand{\past}	{ \smash{\overleftarrow {\meassymbol}} }
\newcommand{\pastprime}	{ {\past}^{\prime}}
\newcommand{\Future}	{ \smash{\overrightarrow{\MeasSymbol}} }
\newcommand{\future}	{ \smash{\overrightarrow{\meassymbol}} }
\newcommand{\futureprime}	{ \smash{\future^{\prime}} }
\newcommand{\PastPrime}	    { \smash{\Past^{\prime}} }
\newcommand{\FuturePrime}	{ \smash{\overrightarrow{\meassymbol}^\prime} }
\newcommand{\PastDblPrime}	{ \smash{\overleftarrow{\meassymbol}^{\prime\prime}} }
\newcommand{\FutureDblPrime}{ \smash{\overrightarrow{\meassymbol}^{\prime\prime}} }
\newcommand{\pastL}	{ \smash{\overleftarrow {\meassymbol}{}^L} }
\newcommand{\PastL}	{ \smash{\overleftarrow {\MeasSymbol}{}^L} }
\newcommand{\PastLt}	  { \smash{\overleftarrow {\MeasSymbol}_t^L} }
\newcommand{\PastLLessOne}{ \smash{\overleftarrow {\MeasSymbol}^{L-1}} }
\newcommand{\futureL}	  { \smash{\overrightarrow{\meassymbol}{}^L} }
\newcommand{\FutureL}	  { \smash{\overrightarrow{\MeasSymbol}{}^L} }
\newcommand{\FutureLt}	  { \smash{\overrightarrow{\MeasSymbol}_t^L} }
\newcommand{\FutureLLessOne}{ \smash{\overrightarrow{\MeasSymbol}^{L-1}} }
\newcommand{\pastLprime}	{ \smash{\overleftarrow {\meassymbol}^{L^\prime}} }
\newcommand{\futureLprime}	{ \smash{\overrightarrow{\meassymbol}^{L^\prime}} }
\newcommand{\AllPasts}	    { \smash{\overleftarrow{ {\rm {\bf \MeasSymbol}} } } }
\newcommand{\AllFutures}	{ \smash{\overrightarrow{ {\rm {\bf \MeasSymbol}} } } }
\newcommand{\FutureSet}	{ \smash{\overrightarrow{\bf \MeasSymbol}} }

% Causal states and epsilon-machines
\newcommand{\CausalState}	{ \mathcal{S} }
\newcommand{\CausalStatePrime}	{ {\CausalState}^{\prime}}
\newcommand{\causalstate}	{ \sigma }
\newcommand{\CausalStateSet}	{ \boldsymbol{\CausalState} }
\newcommand{\AlternateState}	{ \mathcal{R} }
\newcommand{\AlternateStatePrime}	{ {\cal R}^{\prime} }
\newcommand{\alternatestate}	{ \rho }
\newcommand{\alternatestateprime}	{ {\rho^{\prime}} }
\newcommand{\AlternateStateSet}	{ \boldsymbol{\AlternateState} }
\newcommand{\PrescientState}	{ \widehat{\AlternateState} }
\newcommand{\prescientstate}	{ \widehat{\alternatestate} }
\newcommand{\PrescientStateSet}	{ \boldsymbol{\PrescientState}}
\newcommand{\CausalEquivalence}	{ {\sim}_{\epsilon} }
\newcommand{\CausalEquivalenceNot}	{ {\not \sim}_{\epsilon}}

\newcommand{\NonCausalEquivalence}	{ {\sim}_{\eta} }
\newcommand{\NextObservable}	{ {\smash{\overrightarrow {\MeasSymbol}}^1} }
\newcommand{\LastObservable}	{ {\smash{\overleftarrow {\MeasSymbol}}^1} }
%\newcommand{\Prob}		{ {\rm P}}
\newcommand{\Prob}      {\Pr} % use standard command
\newcommand{\ProbAnd}	{ {,\;} }
\newcommand{\LLimit}	{ {L \rightarrow \infty}}
\newcommand{\Cmu}		{C_\mu}
\newcommand{\hmu}		{h_\mu}
\newcommand{\EE}		{{\bf E}}
\newcommand{\TI}		{{\bf T}}
\newcommand{\SI}        {{\bf S}}
\newcommand{\Measurable}{{\bf \mu}}

% Process Crypticity
\newcommand{\PC}		{\chi}

% Causal Irreversibility
\newcommand{\CI}		{\Xi}
\newcommand{\ReverseMap}	{r}
\newcommand{\ForwardMap}	{f}

% Abbreviations from IB:
% None that aren't already in CMPPSS

% Abbreviations from Extensive Estimation:
\newcommand{\EstCausalState}	{\widehat{\CausalState}}
\newcommand{\estcausalstate}	{\widehat{\causalstate}}
\newcommand{\EstCausalStateSet}	{\boldsymbol{\EstCausalState}}
\newcommand{\EstCausalFunc}	{\widehat{\epsilon}}
\newcommand{\EstCmu}		{\widehat{\Cmu}}
\newcommand{\PastLOne}	{{\Past}^{L+1}}
\newcommand{\pastLOne}	{{\past}^{L+1}}

% Abbreviations from $\epsilon$-Transducers:
\newcommand{\InAlphabet}	{ \mathcal{A}}
\newcommand{\insymbol}		{ a}
\newcommand{\OutAlphabet}	{ \mathcal{B}}
\newcommand{\outsymbol}		{ b}
\newcommand{\InputSimple}	{ X}
\newcommand{\inputsimple}	{ x}
\newcommand{\BottleneckVar}	{\tilde{\InputSimple}}
\newcommand{\bottleneckvar}	{\tilde{\inputsimple}}
\newcommand{\InputSpace}	{ \mathbf{\InputSimple}}
\newcommand{\InputBi}	{ \overleftrightarrow {\InputSimple} }
\newcommand{\inputbi}	{ \overleftrightarrow {\inputsimple} }
\newcommand{\InputPast}	{ \overleftarrow {\InputSimple} }
\newcommand{\inputpast}	{ \overleftarrow {\inputsimple} }
\newcommand{\InputFuture}	{ \overrightarrow {\InputSimple} }
\newcommand{\inputfuture}	{ \overrightarrow {\inputsimple} }
\newcommand{\NextInput}	{ {{\InputFuture}^{1}}}
\newcommand{\NextOutput}	{ {\OutputFuture}^{1}}
\newcommand{\OutputSimple}	{ Y}
\newcommand{\outputsimple}	{ y}
\newcommand{\OutputSpace}	{ \mathbf{\OutputSimple}}
\newcommand{\OutputBi}	{ \overleftrightarrow{\OutputSimple} }
\newcommand{\outputbi}	{ \overleftrightarrow{\outputsimple} }
\newcommand{\OutputPast}	{ \overleftarrow{\OutputSimple} }
\newcommand{\outputpast}	{ \overleftarrow{\outputsimple} }
\newcommand{\OutputFuture}	{ \overrightarrow{\OutputSimple} }
\newcommand{\outputfuture}	{ \overrightarrow{\outputsimple} }
\newcommand{\OutputL}	{ {\OutputFuture}^L}
\newcommand{\outputL}	{ {\outputfuture}^L}
\newcommand{\InputLLessOne}	{ {\InputFuture}^{L-1}}
\newcommand{\inputLlessone}	{ {\inputufutre}^{L-1}}
\newcommand{\OutputPastLLessOne}	{{\OutputPast}^{L-1}_{-1}}
\newcommand{\outputpastLlessone}	{{\outputpast}^{L-1}}
\newcommand{\OutputPastLessOne}	{{\OutputPast}_{-1}}
\newcommand{\outputpastlessone}	{{\outputpast}_{-1}}
\newcommand{\OutputPastL}	{{\OutputPast}^{L}}
\newcommand{\OutputLPlusOne}	{ {\OutputFuture}^{L+1}}
\newcommand{\outputLplusone}	{ {\outputfutre}^{L+1}}
\newcommand{\InputPastL}	{{\InputPast}^{L}}
\newcommand{\inputpastL}	{{\inputpast}^{L}}
\newcommand{\JointPast}	{{(\InputPast,\OutputPast)}}
\newcommand{\jointpast}	{{(\inputpast,\outputpast)}}
\newcommand{\jointpastone}	{{(\inputpast_1,\outputpast_1)}}
\newcommand{\jointpasttwo}	{{(\inputpast_2,\outputpast_2)}}
\newcommand{\jointpastprime} {{({\inputpast}^{\prime},{\outputpast}^{\prime})}}
\newcommand{\NextJoint}	{{(\NextInput,\NextOutput)}}
\newcommand{\nextjoint}	{{(\insymbol,\outsymbol)}}
\newcommand{\AllInputPasts}	{ { \overleftarrow {\rm \InputSpace}}}
\newcommand{\AllOutputPasts}	{ {\overleftarrow {\rm \OutputSpace}}}
\newcommand{\DetCausalState}	{ {{\cal S}_D }}
\newcommand{\detcausalstate}	{ {{\sigma}_D} }
\newcommand{\DetCausalStateSet}	{ \boldsymbol{{\CausalState}_D}}
\newcommand{\DetCausalEquivalence}	{ {\sim}_{{\epsilon}_{D}}}
\newcommand{\PrescientEquivalence}	{ {\sim}_{\widehat{\eta}}}
\newcommand{\FeedbackCausalState}	{ \mathcal{F}}
\newcommand{\feedbackcausalstate}	{ \phi}
\newcommand{\FeedbackCausalStateSet}	{ \mathbf{\FeedbackCausalState}}
\newcommand{\JointCausalState}		{ \mathcal{J}}
\newcommand{\JointCausalStateSet}	{ \mathbf{\JointCausalState}}
\newcommand{\UtilityFunctional}	{ {\mathcal{L}}}
\newcommand{\NatureState}	{ {\Omega}}
\newcommand{\naturestate}	{ {\omega}}
\newcommand{\NatureStateSpace}	{ {\mathbf{\NatureState}}}
\newcommand{\AnAction}	{ {A}}
\newcommand{\anaction}	{ {a}}
\newcommand{\ActionSpace}	{ {\mathbf{\AnAction}}}

% Abbreviations from RURO:
\newcommand{\InfoGain}[2] { \mathcal{D} \left( {#1} || {#2} \right) }

% Abbreviations from Upper Bound:
\newcommand{\lcm}	{{\rm lcm}}
% Double-check that this isn't in the math set already!

% Abbreviations from Emergence in Space
\newcommand{\ProcessAlphabet}	{\MeasAlphabet}
\newcommand{\ProbEst}			{ {\widehat{\Prob}_N}}
\newcommand{\STRegion}			{ {\mathrm K}}
\newcommand{\STRegionVariable}		{ K}
\newcommand{\stregionvariable}		{ k}
\newcommand{\GlobalPast}		{ \overleftarrow{G}} 
\newcommand{\globalpast}		{ \overleftarrow{g}} 
\newcommand{\GlobalFuture}		{ \overrightarrow{G}}
\newcommand{\globalfuture}		{ \overrightarrow{g}}
\newcommand{\GlobalState}		{ \mathcal{G}}
\newcommand{\globalstate}		{ \gamma}
\newcommand{\GlobalStateSet}		{ {\mathbf \GlobalState}}
\newcommand{\LocalPast}			{ \overleftarrow{L}} 
\newcommand{\localpast}			{ \overleftarrow{l}}
\newcommand{\AllLocalPasts}		{ \mathbf{\LocalPast}}
\newcommand{\LocalPastRegion}		{ \overleftarrow{\mathrm L}}
\newcommand{\LocalFuture}		{ \overrightarrow{L}}
\newcommand{\localfuture}		{ \overrightarrow{l}}
\newcommand{\LocalFutureRegion}		{ \overrightarrow{\mathrm L}}
\newcommand{\LocalState}		{ \mathcal{L}}
\newcommand{\localstate}		{ \lambda}
\newcommand{\LocalStateSet}		{ {\mathbf \LocalState}}
\newcommand{\PatchPast}			{ \overleftarrow{P}}
\newcommand{\patchpast}			{ \overleftarrow{p}}
\newcommand{\PatchPastRegion}		{ \overleftarrow{\mathrm P}}
\newcommand{\PatchFuture}		{ \overrightarrow{P}}
\newcommand{\patchfuture}		{ \overrightarrow{p}}
\newcommand{\PatchFutureRegion}		{ \overrightarrow{\mathrm P}}
\newcommand{\PatchState}		{ \mathcal{P}}
\newcommand{\patchstate}		{ \pi}
\newcommand{\PatchStateSet}		{ {\mathbf \PatchState}}
\newcommand{\LocalStatesInPatch}	{\vec{\LocalState}}
\newcommand{\localstatesinpatch}	{\vec{\localstate}}
\newcommand{\PointInstantX}		{ {\mathbf x}}
% Galles's original LaTeX for the cond. indep. symbol follows:
\newcommand{\compos}{\mbox{$~\underline{~\parallel~}~$}}
\newcommand{\ncompos}{\not\hspace{-.15in}\compos}
\newcommand{\indep}			{ \rotatebox{90}{$\models$}}
\newcommand{\nindep}	{\not\hspace{-.05in}\indep}
\newcommand{\LocalEE}	{{\EE}^{loc}}
\newcommand{\EEDensity}	{\overline{\LocalEE}}
\newcommand{\LocalCmu}	{{\Cmu}^{loc}}
\newcommand{\CmuDensity}	{\overline{\LocalCmu}}

%%%%%%%%%%% added by sasa
\newcommand{\FinPast}[1]	{ \overleftarrow {\MeasSymbol} \stackrel{{#1}}{}}
\newcommand{\finpast}[1]  	{ \overleftarrow {\meassymbol}  \stackrel{{#1}}{}}
\newcommand{\FinFuture}[1]		{ \overrightarrow{\MeasSymbol} \stackrel{{#1}}{}}
\newcommand{\finfuture}[1]		{ \overrightarrow{\meassymbol} \stackrel{{#1}}{}}

\newcommand{\Partition}	{ \AlternateState }
\newcommand{\partitionstate}	{ \alternatestate }
\newcommand{\PartitionSet}	{ \AlternateStateSet }
\newcommand{\Fdet}   { F_{\rm det} }

\newcommand{\Dkl}[2] { D_{\rm KL} \left( {#1} || {#2} \right) }

\newcommand{\Period}	{p}

% To take into account time direction
\newcommand{\forward}{+}
\newcommand{\reverse}{-}
%\newcommand{\forwardreverse}{\:\!\diamond} % \pm
\newcommand{\forwardreverse}{\pm} % \pm
\newcommand{\FutureProcess}	{ {\Process}^{\forward} }
\newcommand{\PastProcess}	{ {\Process}^{\reverse} }
\newcommand{\FutureCausalState}	{ {\CausalState}^{\forward} }
\newcommand{\futurecausalstate}	{ \sigma^{\forward} }
\newcommand{\altfuturecausalstate}	{ \sigma^{\forward\prime} }
\newcommand{\PastCausalState}	{ {\CausalState}^{\reverse} }
\newcommand{\pastcausalstate}	{ \sigma^{\reverse} }
\newcommand{\BiCausalState}		{ {\CausalState}^{\forwardreverse} }
\newcommand{\bicausalstate}		{ {\sigma}^{\forwardreverse} }
\newcommand{\FutureCausalStateSet}	{ {\CausalStateSet}^{\forward} }
\newcommand{\PastCausalStateSet}	{ {\CausalStateSet}^{\reverse} }
\newcommand{\BiCausalStateSet}	{ {\CausalStateSet}^{\forwardreverse} }
\newcommand{\eMachine}	{ M }
\newcommand{\FutureEM}	{ {\eMachine}^{\forward} }
\newcommand{\PastEM}	{ {\eMachine}^{\reverse} }
\newcommand{\BiEM}		{ {\eMachine}^{\forwardreverse} }
\newcommand{\BiEquiv}	{ {\sim}^{\forwardreverse} }
\newcommand{\Futurehmu}	{ h_\mu^{\forward} }
\newcommand{\Pasthmu}	{ h_\mu^{\reverse} }
\newcommand{\FutureCmu}	{ C_\mu^{\forward} }
\newcommand{\PastCmu}	{ C_\mu^{\reverse} }
\newcommand{\BiCmu}		{ C_\mu^{\forwardreverse} }
\newcommand{\FutureEps}	{ \epsilon^{\forward} }
\newcommand{\PastEps}	{ \epsilon^{\reverse} }
\newcommand{\BiEps}	{ \epsilon^{\forwardreverse} }
\newcommand{\FutureSim}	{ \sim^{\forward} }
\newcommand{\PastSim}	{ \sim^{\reverse} }
\newcommand{\FuturePC}	{\PC^{\forward}}
\newcommand{\PastPC}		{\PC^{\reverse}}
\newcommand{\BiPC}      {\PC^{\forwardreverse}}

% Used arrows for awhile, more or less confusing?
%\newcommand{\FutureCausalState}	{ \overrightarrow{\CausalState} }
%\newcommand{\PastCausalState}	{ \overleftarrow{\CausalState} }
%\newcommand{\eMachine}	{ M }
%\newcommand{\FutureEM}	{ \overrightarrow{\eMachine} }
%\newcommand{\PastEM}	{ \overleftarrow{\eMachine} }
%\newcommand{\FutureCmu}	{ \overrightarrow{\Cmu} }
%\newcommand{\PastCmu}	{ \overleftarrow{\Cmu} }

%% time-reversing and mixed state presentation operators
\newcommand{\TR}{\mathcal{T}}
\newcommand{\MSP}{\mathcal{U}}
\newcommand{\one}{\mathbf{1}}

%% added for RURO2
%% Note that it will always be typesett in math mode.
%% Example usage:
%%    The 3-block entropy is \BE[t]{3}.
%%    The 3-block-state entropy from $t=0$ is $\BSE{3}$.
%%
%% The optional argument specifies the start time, defaulting to t=0.
%% The mandatory argument specifies the length of the block (not the stop time).
\newcommand{\lastindex}[2]{
  \edef\tempa{0}
  \edef\tempb{#2}
  \ifx\tempa\tempb
    % if the length is 0, then the final time equals the start time
    \edef\tempc{#1}
  \else
    % if the start time is different from zero, then we show the sum
    \edef\tempa{0}
    \edef\tempb{#1}
    \ifx\tempa\tempb
      \edef\tempc{#2}
    \else
      \edef\tempc{#1+#2}
    \fi
  \fi
  \tempc
}
\newcommand{\BE}[2][0]{
  \ensuremath{H[\MeasSymbol_{#1}^{#2}]}
}
\newcommand{\BSE}[2][0]{
  \ensuremath{H[\MeasSymbol_{#1}^{#2},\CausalState_{\lastindex{#1}{#2}}]}%
}
\newcommand{\SBE}[2][0]{
  \ensuremath{H[\CausalState_{#1},\MeasSymbol_{#1}^{#2}]}
}
\newcommand{\SBSE}[2][0]{
  \ensuremath{H[\CausalState_{#1},\MeasSymbol_{#1}^{#2},
                \CausalState_{\lastindex{#1}{#2}}]}
}
\newcommand{\AltSBE}[2][0]{
  \ensuremath{H[\AlternateState_{#1},\MeasSymbol_{#1}^{#2}]}
}
\newcommand{\AltBSE}[2][0]{
  \ensuremath{H[\MeasSymbol_{#1}^{#2},\AlternateState_{\lastindex{#1}{#2}}]}
}
\newcommand{\AltSBSE}[2][0]{
  \ensuremath{H[\AlternateState_{#1},\MeasSymbol_{#1}^{#2},
                \AlternateState_{\lastindex{#1}{#2}}]}
}

\newcommand{\GI}{\varphi}
\newcommand{\OI}{\zeta}
% cryptic, gauge, oracular, Markov, and synchronization
\newcommand{\COrder}{k_{\PC}}
\newcommand{\GOrder}{k_{\GI}}
\newcommand{\OOrder}{k_{\OI}}
\newcommand{\MOrder}{R}
\newcommand{\SOrder}{k_{\SI}}

%% Added to standardize the order of causal states
%% Example usage:
%%    $\BiCmu = H[\CSjoint] = H[\CSjoint[,]]$
%%    $\EE = I[\CSjoint[:]] = I[\CSjoint[;]]$
\newcommand{\CSjoint}[1][,]{
   \edef\tempa{:}
   \edef\tempb{#1}
   \ifx\tempa\tempb
      % arg1 was a colon, insert a small negative space
      \ensuremath{\FutureCausalState\!#1\PastCausalState}
   \else
      % arg1 was not a colon, usually a comma or semicolon
      \ensuremath{\FutureCausalState#1\PastCausalState}
   \fi
}

%% (cje)
%% Provide a command \ifpm which is true when \pm 
%% is meant to be understood as "+ or -", opposed
%% to "bidirectional". This differs from the usage 
%% in TBA.  Essentially, if \forwardreverse is defined
%% as something different, perhaps a \diamond, then
%% we are free to interpret \pm as "+ or -".
\newif\ifpm 
\edef\tempa{\forwardreverse}
\edef\tempb{\pm}
\ifx\tempa\tempb
   \pmfalse
\else
   \pmtrue  
\fi

%% Macros for krorder
\newcommand{\MeasSymbols}[2] {\MeasSymbol_{#1:#2}}
\newcommand{\Lsync} {\mathcal{L}_{\mathrm{sync}}}


